\input{../../slides_preamb.tex}


\subtitle{Introduction (and why Python?)}

\author{John Stachurski}


\date{September 2022}


\begin{document}

\begin{frame}
  \titlepage
\end{frame}





\section{Introduction}



\begin{frame}
    %\frametitle{Introduction}

    \textbf{Today}

    \begin{itemize}
        \item Introduction to scientific computing with Python
        \vspace{0.5em}
        \vspace{0.5em}
        \item Fixed points and job search 
        \vspace{0.5em}
        \vspace{0.5em}
        \item Dynamic programming: theory and algorithms
        \vspace{0.5em}
        \vspace{0.5em}
        \item Parallelization on the GPU 
    \end{itemize}

\end{frame}




\begin{frame}


    Assumptions:

    \begin{itemize}
        \item You have read lectures 1-3 at
            %
            \begin{center}
                \texttt{https://python-programming.quantecon.org/intro.html}
            \end{center}
        \vspace{0.3em}
        \item some basic familiarity with programming
    \end{itemize}

    \begin{itemize}
        \item \url{https://github.com/QuantEcon/rse_comp_econ_2022}
    \end{itemize}


\end{frame}




\section{Overview}


\begin{frame}
    \frametitle{Background --- Language Types}
    
    \navy{Proprietary} 
    %
    \begin{itemize}
        \item Excel
        \item MATLAB
        \item STATA, etc.
    \end{itemize}
    

    \vspace{0.5em}
    \vspace{0.5em}
    \navy{Open Source} 
    
    \begin{itemize}
        \item Python
        \item Julia
        \item R
    \end{itemize}


    \begin{center}
        closed and stable vs open and fast moving
    \end{center}

\end{frame}





\begin{frame}
    \frametitle{Background --- Language Types}
    
    \navy{Low level } 
    
    \begin{itemize}
        \item C/C++
        \item Fortran
        \item Rust
    \end{itemize}

    \vspace{1em}

    \navy{High level } 

    \begin{itemize}
        \item Python
        \item Ruby
        \item TypeScript
    \end{itemize}

\end{frame}




\begin{frame}[fragile]

    Low level languages give us fine grained control 
    
    \Eg \brown{1 + 1} in assembly

    {\small
    \begin{minted}{as}
pushq   %rbp
movq    %rsp, %rbp
movl    $1, -12(%rbp)
movl    $1, -8(%rbp)
movl    -12(%rbp), %edx
movl    -8(%rbp), %eax
addl    %edx, %eax
movl    %eax, -4(%rbp)
movl    -4(%rbp), %eax
popq    %rbp
    \end{minted}
    }



\end{frame}


\begin{frame}
    
    \navy{High level languages} give us abstraction, automation, etc.

\end{frame}



\begin{frame}[fragile]

    \Eg Reading from a file in Python
    
    \begin{minted}{python}
    data_file = open("data.txt")
    for line in data_file:
        print(line.capitalize()) 
    data_file.close()
    \end{minted}

\end{frame}


\begin{frame}
    \frametitle{Trade-Offs}

    \begin{figure}
       \begin{center}
        \scalebox{.36}{\includegraphics{tradeoff.pdf}}
       \end{center}
    \end{figure}

\end{frame}



\begin{frame}[fragile]
    \frametitle{But what about scientific computing?}
    
    \navy{Requirements}

    \begin{itemize}
        \item \underline{\green{Productive}} --- easy to read, write, debug, explore
            \vspace{0.4em}
            \vspace{0.4em}
            \vspace{0.4em}
        \item \underline{\green{Fast}} computations
    \end{itemize}

\end{frame}




\begin{frame}
    \frametitle{Trade-Offs}

    \begin{figure}
       \begin{center}
        \scalebox{.36}{\includegraphics{tradeoff2.pdf}}
       \end{center}
    \end{figure}

\end{frame}


\begin{frame}
    \frametitle{Trade-Offs}

    \begin{figure}
       \begin{center}
        \scalebox{.36}{\includegraphics{tradeoff3.pdf}}
       \end{center}
    \end{figure}

\end{frame}


\begin{frame}
    \frametitle{Trade-Offs}

    \begin{figure}
       \begin{center}
        \scalebox{.36}{\includegraphics{tradeoff4.pdf}}
       \end{center}
    \end{figure}

\end{frame}




\section{Trends}

\begin{frame}
    \frametitle{Trend 1: Parallelization}

    CPU frequency (clock speed) growth is slowing

    \begin{figure}
       \begin{center}
        \scalebox{.22}{\includegraphics{processor_clock.png}}
       \end{center}
    \end{figure}

\end{frame}


\begin{frame}
    
    Chip makers have responded by developing multi-core processors

    \begin{figure}
       \begin{center}
        \scalebox{.2}{\includegraphics{dual_core.png}}
       \end{center}
    \end{figure}

    Source: Wikipedia


\end{frame}


\begin{frame}

    \navy{GPUs / ASICs} are also becoming increasingly important


    \begin{figure}
       \begin{center}
        \scalebox{.16}{\includegraphics{gpu.jpg}}
       \end{center}
    \end{figure}

    \vspace{0.5em}

    Applications: machine learning, deep learning, etc.
    

\end{frame}



\begin{frame}
    \frametitle{Trend 2: Distributed Computing}
    
    Advantages: 
    %
    \begin{itemize}
        \item run code on big machines we don't have to buy
        \vspace{0.5em}
        \item customized execution environments
        \vspace{0.5em}
        \item circumvent annoying internal IT departments
    \end{itemize}

    \vspace{0.5em}

    Options:
    %
    \begin{itemize}
        \item University machines
            \vspace{0.5em}
        \item AWS 
            \vspace{0.5em}
        \item Google Colab, etc.
    \end{itemize}

\end{frame}




\section{Which Language?}


\begin{frame}
    \frametitle{Which Language}


    How about R?
    \vspace{0.5em}

    \begin{itemize}
        \item Specialized to statistics
            \vspace{0.5em}
        \item Easy to learn, well designed
            \vspace{0.5em}
        \item Huge range of estimation routines
            \vspace{0.5em}
        \item Significant demand for R programmers
            \vspace{0.5em}
        \item Popular in academia 
    \end{itemize}

\end{frame}


\begin{frame}
    
    However loosing ground to Python
            \vspace{0.5em}
            \vspace{0.5em}
            \vspace{0.5em}

    \Eg Chris Wiggins, Chief Data Scientist at The New York Times:

            \vspace{0.5em}
     ``Python has gotten sufficiently weapons grade that we don't descend into
     R anymore. Sorry, R people. I used to be one of you but we no longer
     descend into R.''


\end{frame}


\begin{frame}
    \frametitle{Julia}

    Pros:
    
    \begin{itemize}
        \item Fast and elegant
            \vspace{0.5em}
        \item Many scientific routines
            \vspace{0.5em}
        \item Julia is written in Julia
    \end{itemize}

            \vspace{0.5em}
            \vspace{0.5em}
            \vspace{0.5em}
    Cons:
    
    \begin{itemize}
        \item Some stability issues
            \vspace{0.5em}
        \item Failing to achieve rapid growth
    \end{itemize}

\end{frame}




\begin{frame}
    \frametitle{Python}
    
    \begin{itemize}
        \item Easy to learn, well designed
            \vspace{0.5em}
        \item Massive scientific ecosystem
            \vspace{0.5em}
        \item Heavily supported by big players
            \vspace{0.5em}
        \item Open source
            \vspace{0.5em}
        \item Huge demand for tech-savvy Python programmers
    \end{itemize}

\end{frame}




\begin{frame}
    \frametitle{Scientific Computing}
    
    Python has strong tools in vectorization / JIT compilation /
    parallelization / visualization / etc.

    Examples:

    \begin{itemize}
        \item SciPy, NumPy, Matplotlib, pandas
            \vspace{0.5em}
        \item Numba (JIT compilation, multithreading)
            \vspace{0.5em}
        \item Tensorflow, PyTorch (machine learning, AI)
            \vspace{0.5em}
        \item JAX (JIT compilation, parallelization), etc., etc.
    \end{itemize}

\end{frame}

\begin{frame}
    

    Popularity, others vs one Python library (pandas)

    \begin{figure}
       \begin{center}
        \scalebox{.4}{\includegraphics{python_vs_rest.png}}
       \end{center}
    \end{figure}


\end{frame}








\section{Set Up}







\begin{frame}
    \frametitle{Downloads / Installation / Troubleshooting}

    \green{Install Python + Scientific Libs (Optional!)}
    
    \begin{itemize}
        \item Install Anaconda from {\footnotesize \url{https://www.anaconda.com/}}
        \vspace{1em}
            \begin{itemize}
                \item Select latest version 
                \item For your OS
                \item Say ``yes'' at prompts
            \end{itemize}
        \vspace{1em}
        \item Not plain vanilla Python
    \end{itemize}


    \vspace{1em}

    \green{Remote options}

    \begin{itemize}
        \item \url{https://colab.research.google.com}
        \item \url{https://www.pythonanywhere.com/}
    \end{itemize}


\end{frame}



\begin{frame}
    \frametitle{Jupyter Notebooks}

    A browser based interface to Python / Julia / R / etc.


    \vspace{2em}

    \begin{itemize}
        \item Search for \texttt{jupyter notebook}
    \end{itemize}

    \vspace{2em}

    Useful for:

    \begin{itemize}
        \item getting started
        \item exploring ideas
    \end{itemize}





\end{frame}



\begin{frame}
    \frametitle{Working with Notebooks}

    \begin{itemize}
        \item Entry and execution
    \vspace{1em}
        \item Markdown
    \vspace{1em}
        \item Getting help
    \vspace{1em}
        \item Copy paste
    \vspace{1em}
        \item Edit and command mode
    \end{itemize}

\end{frame}





\end{document}


